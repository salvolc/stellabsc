\chapter{Motivation for a Search for Vector-Like Quarks}




\section{Short Overview of the Standard Model of Particle Physics}
Particle physics deals with elementary particles without any substructure and bound states of the elemantary particles forming other particles.
Moreover, the interactions between the particles  are studied.
Table \ref{standardmodel} shows the different elementary particles the Standard Model consists of.

\begin{table}
\centering
%\setlength{\tabcolsep}{3cm}
\begin{tabular}{|c|c c c|c|} 
\hline
 & $1^{st}$ generation & $2^{nd}$ generation & $3^{rd}$ generation &  Bosons \\
\hline
\hline
&	&	&	& \small{Vector-bosons}	\\
Quarks &  $\begin{matrix} u \\ \\ d \end{matrix}$ & $\begin{matrix} c \\ \\ s \end{matrix}$ & $\begin{matrix} t \\ \\ b \end{matrix}$ & $\gamma$ \\
& & & & g  \\
&	&	&	&	\\
& &  &  & $W^{\pm}$, $Z^{0}$ \\
Leptons & $\begin{matrix} \nu_{e} \\ \\ e \end{matrix}$ & $\begin{matrix} \nu_{\mu} \\ \\ \mu \end{matrix}$ & $\begin{matrix} \nu_{\tau} \\ \\ \tau \end{matrix}$ & \small{Scalar boson}  \\
&	&	&	&   H \\
&	&	&	&     \\
 \hline
\end{tabular}
\caption{Elemantary particles of the Standard Model.}
\label{standardmodel}
\end{table}



Both leptons and quarks are fermions with spin $s= \frac{1}{2}$. 
There are twelve fermions, with one anti-particle for each fermion, which are divided in three generations.     
For the leptons there is a lepton with charge $q = -1$ (electron e, muon $\mu$, tau $\tau$) and a corresponding neutrino without electric charge (electron-neutrino $\nu_{e}$, muon-neutrino $\nu_{\mu}$, tau-neutrino $\nu_{\tau}$) in each genaration.
The mass of charged leptons increases with the generation in which it is located and the neutrinos are supposed to be massless.\\
Quarks are elementary particles which cannot be observed in unbound states.
The quarks can be divided in up-type and down-types with charge $q = +\frac{2}{3}$ for the up- and $q = -\frac{1}{3}$ for the down-types.
Up-quark u, charm-quark c and top-quark t are up-types while down-quark d, strange-quark s and bottom-quark b are down-type quarks. 
The mass of the quarks increases with the generation as it was mentioned for the leptons.
The top-quark is of peculiar interest because it is the heaviest particle of the Standard Model and unlike every other quark does not occur in bound states.
Furthermore the Standard Model includes particles with spin 1 which are the gauge bosons.
They include the photon $\gamma$,the gluons g, the W bosons $W^{\pm}$ and the Z boson $Z^{0}$.
The gauge bosons are the mediators of the fundamental forces. 
The photon $\gamma$ mediates the electromagnetic, the gluons the strong  and the $W^{\pm}$ and $Z^{0}$ the weak force.  
Fermions and particles, which are bound states of the fermions, interact via the fundamental forces.
There are different characteristics particles have to fulfill to participate in the different interactions.
Particles with electric charge can interact via the electromagnetic force.
The strong interaction occurs between particles with colour charge, which are quarks and gluons themselves.
The weak force is mediated between particles with weak charge having a left-handed chirality.\\
The $\gamma$ and the gluons are massless, while the $W^{\pm}$ and $Z^{0}$ have a mass. 
W bosons are the only gauge bosons which have electric charge with +1 for the $W^{+}$ and -1 for the $W^{-}$.\\
Finally there is a scalar boson with spin 0, the higgs boson H.
The higgs boson carries mass but no electric charge.
It was predicted  by the higgs mechanism.
The mechanism explains the existence of mass for the elementary particles of the Standard Model.




\section{Beyond Standard Model Physics: Vector-like Quarks}
Vector-like quarks \cite{handbook} are part of  theories beyond the Standard Model. 
They are for example a necessary part of theories predicting the higgs as pseudo-Goldstone boson to explain why the mass of the higgs is as light as observed \cite{Th1,Th2,Th3}. 
Furthermore, they are part of the partial-compositeness theory of flavour in which they have an appearance as fermion resonances \cite{Th4,Th5}.
These ideas are used for example in little Higgs and composite Higgs models.\\
In addition to that, if the vector-like quarks will exist, they would offer the possibility to allow tree-level flavour-changing neutral currents and to break the GIM mechanism \cite{GIM}.
Moreover, new origins of CP violation would arise from the existence of vector-like quarks \cite{CP1,CP2}.\\
In the theories mentioned before, vector-like quarks  are predicted as fermions with spin $s = \frac{1}{2}$.
There are four types of vector-like quarks which are assumed to couple predominantly to the third generation Standard Model quarks.\\
The different vector-like quarks may appear in seven multiplets of which two are singulets, three doublets and two triplets.
Table \ref{vectorlikequarks} lists the different vector-like quark flavors with their specific charge and the possible multiplets they can be assigned too.

\begin{table}
\centering
\begin{tabular}{|c|c||c|} 
\hline
Flavor& Charge & Arrangement in multiplets\\
\hline
  & 	           & \multirow{3}{*}{$T^{0}_{L,R}$ , $B^{0}_{L,R}$ (singlets)}\\
T & $+\sfrac{2}{3}$ & \multirow{4}{*}{(X $T^{0})_{L,R}$,  ($T^{0} B^{0})_{L,R}$, ($B^{0} Y)_{L,R}$ (doublets)} \\
B & $-\sfrac{1}{3}$ & \\
Y & $+\sfrac{5}{3}$ & \multirow{3}{*}{(X $T^{0} B^{0} )_{L,R}$,  ($T^{0} B^{0} Y)_{L,R}$ (triplets)} \\  
X & $-\sfrac{4}{3}$ & \\
  &                &\\
\hline
\end{tabular}
\caption{This table contains the different types of vector-like quarks with their specific charge and the possible arrangement in multiplets. 
The zero superscript is used to denote the weak eigenstates and separate them from the mass eigenstates.
For the multiplets containing X and Y the mass and weak eigenstates are the same. }
\label{vectorlikequarks}
\end{table}

Unlike the Standard Model quarks coupling with a (V-A), which means a vector minus axialvector coupling structure regarding the weak interaction, the vector-like quarks have pure vector couplings.
Therfore, also vector-like quarks with right-handed chirality couple to the weak force.
Since the lagrangian describing the properties of the vector-like quarks is invariant under SU(2) transformations, explicit mass terms in the lagrangian are not forbidden because they do not break the local gauge invariance.
Hence, the vector-like quarks do not receive their masses from the higgs mechanism in contrast to the Standard Model quarks.\\
Because of the almost equal kinematic for the vector-like top and bottom quarks arranged in singulets, doublets and triplets the multiplets can be used similary for optimization studies \cite{8tevanalysis}.
Therfore mostly only singulets are considered .\\
Possible decay channels for the vector-like quarks include a combination of the vector bosons $W^{\pm}$, $Z^{0}$ or the higgs and a third generation quarks so that conservation of charge is guaranteed.\\
Vector-like quarks can be both pair and singly produced.
The main process for the pair production is the gluon fusion via the strong interaction.
The Feynman diagrams for the pair production process with the specific vector-like quark decay topology TT \texorpdfstring{$\longrightarrow$}~ZtWb and BB \texorpdfstring{$\longrightarrow$}~ZbWt are presented in figure \ref{feynmantts} for the vector-like top and \ref{feynmanbbs} for the vector-like bottom quark.
\begin{figure}[h!]
\centering
\includegraphics[width=11cm]{figures/TTS.pdf}
\caption{Feynman diagram for the vector-like top pair production process and the TT \texorpdfstring{$\longrightarrow$}~ZtWb decay channel.}
\label{feynmantts}
\end{figure}

\newpage

\begin{figure}[h!]
\centering
\includegraphics[width=11cm]{figures/BBS.pdf}
\caption{Feynman diagram for the vector-like bottom pair production process and the BB \texorpdfstring{$\longrightarrow$}~ZbWt decay channel.}
\label{feynmanbbs}
\end{figure}

In the following, the dilepton channel is considered, therefore all particles except the Z boson decay hadronically. 
The presented decay topology contains tree-level flavour-changing neutral currents.\\
Results from previous searches for pair production of vector-like T and B  in the di- and trilepton channel set mass limits for the vector-like top and bottom quark masses. 
The limits for the singlets are $m_{T} >$ \SI{780}{GeV} for the vector-like top and $m_{B} >$ \SI{830}{GeV} for the vector-like bottom quark \cite{masslimit}.  
This limits are determined with a confidence level of $95 \%$.







 










