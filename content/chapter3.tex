% kapitel3.tex
\chapter{Monte-Carlo Samples}

\section{Background Processes}
\label{samples}
In order to perform an unbiased analysis in particle physics, the analysis strategy is studied with simulated samples and only later on compared to data.
In this analysis, only simulated data are considered.
The simulated background samples contain background processes generated with Monte Carlo simulations.
For the background,  Monte Carlos samples uses random number generators in combination with branching ratios predicted in the Standard Model to simulate processes.
The Monte Carlo samples used in this analysis were passed through a full detector simulation of the ATLAS detector \cite{simulation1, simulation2}.\\
The background is divided into five parts, considering the dominant background processes separately and combining the other backgrounds in one category.
There are the Z+jets backgrounds, which include Z+bottom, Z+charm and Z+light processes.
The Z+bottom process is composed of a Z candidate and a number of b-jets, both are produced at the same time.
The other Z+jets backgrounds are composed similarly, with a number of charm-jets for Z+charm and a number of light-jets (strange-,up-,down-jets).
Moreover, there is the \ttbar{} background, which includes a produced top-quark pair, and the category ``other background'', which contains for example diboson processes, $t \bar{t} e^{+}\!e^{-}$ and $t \bar{t} \mu^{+}\!\mu^{-}$ processes.
The diboson processes include a produced pair of vector bosons (WW, ZW, ZZ) and the $t \bar{t} e^{+}\!e^{-}$ respectively $t \bar{t} \mu^{+}\!\mu^{-}$ processes a associated production of a top-quark pair and a pair of electrons or muons.
Since the different processes and the signal can lead to simular signatures in the detector, caused by partially equal final state products, the processes are considered as main background. \\
The various processes are simulated with different event generators.
Event generators simulate the processes starting from the pp-collision and seperate them into the following stages:
Production of heavy particles, radiation of lighter particles (gluons, photons), decay of heavy unstable particles, hadronization and the decays of the particles resulting from the hadronization into long-lived particles. 
The resulting particles finally enter the detector simulation.
There are event generators, which only generate until the hadronization process and therefore shower programs are added.\\
To simulate the Z+jets samples, the event generator Sherpa 2.2 is used.
Sherpa \cite{Sherpa} performs both the simulation of processes until the hadronization and beyond the hadronization.
Sherpa simulates at leading order precision in $\alpha_{s}$. 
For the \ttbar{} background processes, the generator Powheg is utilized with Pythia 8.1 \cite{Pythia} as shower program.
Powheg \cite{Powheg} is a next-to leading order event generator.
For the diboson background, there are both simulated samples with Sherpa 2.2 and Powheg + Pythia 8.1.
The processes $t \bar{t} e^{+}\!e^{-}$  and $t \bar{t} \mu^{+}\!\mu^{-}$ are simulated with Madgraph as event generator and Pythia 8.1 as shower program.
Madgraph \cite{Madgraph} is a event generator which simulates at leading order.\\
In the analysis the samples are normalized to the value of the integrated luminosity $\int{L} \mathrm{d}t = 3.2 \mathrm{fb}^{-1}$ to provide the possibility of comparing the simulated data with the measured data.\\
In the following analysis, the weighted number of events denote the expected number of events considering the integrated luminosity $\int{L} \mathrm{d}t = 3.2 \mathrm{fb}^{-1}$.


\section{Signal Processes}
The signal also includes processes generated from Monte Carlo simulations. 
In this analysis, both vector-like top and bottom quark are considered to be arranged in the singlet.
The processes are referred to as TTS for the vector-like top quark pair production and BBS for the vector-like bottom quark pair production.
Both TTS and BBS are simulated with the event generator Protos and Pythia 8.1 as shower program.
Protos \cite{Protos} is a leading order event generator.\\
As mentioned in section \ref{samples} for the backround, the samples for the signal are also normalized to the value of the integrated luminosity $\int{L} \mathrm{d}t = 3.2 \mathrm{fb}^{-1}$.
Therefore the weighted number of events also represents the expected number of events for the integrated luminosity $\int{L} \mathrm{d}t = 3.2 \mathrm{fb}^{-1}$.
