% kapitel3.tex
\chapter{Monte-Carlo Samples}

\section{Background Processes}
\label{samples}
In order to predict the Standard Model backgrounds for this analysis simulated samples are used. 
The samples pass through a full detector simulation.
The background is divided into five parts.
There are the Z+jets backgrounds, which includes Z+bottom, Z+charm and Z+light processes, the \ttbar{} background and the category \textit{other background}, which contains for example diboson (WW, ZW, ZZ) processes, ttee and tt$\mu \mu$ processes.
The background processes are choosed as background because they can lead to simular signatures, as compared with the signal, caused by the partially equal final state products.\\
The various processes are simulated with different event generators.
Event generator simulate the processes starting from the pp-collusion and seperates the processes into different stages.
Therby production of heavy particles, radiating of lighter particles (gluons, photons), decay of heavy unstable particles, hadronization and the decays of the particles resulting from the hadronisations into long-lived particles, which finally go into the detector, are considered.
There are event generations, which only generate till the hadronization process and therfore shower programs are added.
To simulate the Z+jets samples the event genarator Sherpa 2.2 is used.
Sherpa \cite{Sherpa} performs both simulating processes till the hadronization and beyond the hadronization at leading order precision. 
For the \ttbar{} background processes the generator Powheg is utilized with Pythia 8.1 \cite{Pythia} as shower program.
Powheg \cite{Powheg} is a next-to leading order event generator.
For the diboson background there are both simulated samples with Sherpa 2.2 and Powheg \& Pythia 8.1.
The ttee and $\mathrm{tt \mu \mu}$ processes are simulated with Madgraph  and Pythia 8.1 as shower program.
Madgraph \cite{Madgraph} is a eventgenerator which simulates at leading order.\\
The used Monte-Carlo samples can be normalized to the value of the integrated luminosity by applying weights to the samples.
Moreover weights for example for pile up can be applied.\\
The weights have an influence on the number of events. 


\section{Signal Processes}
The signal processes are simulated as mentioned for the background processes.
Both vector-like top and bottom quark are considered to appear in the singlet.
The signal processes are simulated with Protos and Pythia 8.1 as shower program.
Protos \cite{Protos} is a leading order event generator.\\
For the signal processes the same weights can be applied as for the background processes.
