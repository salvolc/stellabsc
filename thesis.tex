%%%%%%%%%%%%%%%%%%%%%%%%%%%%%%%%%%%%%%%%%%%%%%%%%%%%%%%%%%%%%%%%%%%%%%%%%%%%%%%%
%%%%%%%%%%%%%%%%%%   Vorlage für eine Abschlussarbeit   %%%%%%%%%%%%%%%%%%%%%%%%
%%%%%%%%%%%%%%%%%%%%%%%%%%%%%%%%%%%%%%%%%%%%%%%%%%%%%%%%%%%%%%%%%%%%%%%%%%%%%%%%

% Erstellt von Maximilian Nöthe, <maximilian.noethe@tu-dortmund.de>
% ausgelegt für lualatex und Biblatex mit biber

% Kompilieren mit 
% lualatex dateiname.tex
% biber dateiname.bcf
% lualatex dateiname.tex
% lualatex dateiname.tex
% oder einfach mit:
% make

\documentclass[
  BCOR=12mm,     % 12mm binding corrections, adjust to fit your binding
  parskip=half,  % new paragraphs start with half line vertical space
  open=any,      % chapters start on both odd and even pages
  cleardoublepage=plain,  % no header/footer on blank pages
]{tudothesis}


% Warning, if another latex run is needed
\usepackage[aux]{rerunfilecheck}

% just list chapters and sections in the toc, not subsections or smaller
\setcounter{tocdepth}{1}

%------------------------------------------------------------------------------
%------------------------------ Sprache und Schrift: --------------------------
%------------------------------------------------------------------------------
\usepackage{fontspec}
\defaultfontfeatures{Ligatures=TeX}  % -- becomes en-dash etc.

% german language
\usepackage{polyglossia}
\setdefaultlanguage{english}

% for english abstract and english titles in the toc
\setotherlanguages{english}

% intelligent quotation marks, language and nesting sensitive
\usepackage[autostyle]{csquotes}

% microtypographical features, makes the text look nicer on the small scale
\usepackage{microtype}

%------------------------------------------------------------------------------
%------------------------ Für die Matheumgebung--------------------------------
%------------------------------------------------------------------------------

\usepackage{amsmath}
\usepackage{amssymb}
\usepackage{mathtools}

% Enable Unicode-Math and follow the ISO-Standards for typesetting math
\usepackage[
  math-style=ISO,
  bold-style=ISO,
  sans-style=italic,
  nabla=upright,
  partial=upright,
]{unicode-math}
\setmathfont{Latin Modern Math}

% nice, small fracs for the text with \sfrac{}{}
\usepackage{xfrac}  


%------------------------------------------------------------------------------
%---------------------------- Numbers and Units -------------------------------
%------------------------------------------------------------------------------

\usepackage[
  locale=DE,
  separate-uncertainty=true,
  per-mode=symbol-or-fraction,
]{siunitx}
\sisetup{math-micro=\text{µ},text-micro=µ}

%------------------------------------------------------------------------------
%-------------------------------- tables  -------------------------------------
%------------------------------------------------------------------------------

\usepackage{booktabs}       % stellt \toprule, \midrule, \bottomrule

%------------------------------------------------------------------------------
%-------------------------------- graphics -------------------------------------
%------------------------------------------------------------------------------

\usepackage{graphicx}
\usepackage{grffile}

% allow figures to be placed in the running text by default:
\usepackage{scrhack}
\usepackage{float}
\floatplacement{figure}{htbp}
\floatplacement{table}{htbp}

% keep figures and tables in the section
\usepackage[section, below]{placeins}


%------------------------------------------------------------------------------
%---------------------- customize list environments ---------------------------
%------------------------------------------------------------------------------

\usepackage{enumitem}

%------------------------------------------------------------------------------
%------------------------------ Bibliographie ---------------------------------
%------------------------------------------------------------------------------

\usepackage[
  backend=biber,   % use modern biber backend
  sorting=none,
  autolang=hyphen, % load hyphenation rules for if language of bibentry is not
                   % german, has to be loaded with \setotherlanguages
                   % in the references.bib use langid={en} for english sources
]{biblatex}
\addbibresource{references.bib}  % die Bibliographie einbinden
\DefineBibliographyStrings{german}{andothers = {{et\,al\adddot}}} 

%------------------------------------------------------------------------------
%------------------------------ Sonstiges: ------------------------------------,linkbordercolor=tugreen
%------------------------------------------------------------------------------

\usepackage[pdfusetitle,unicode]{hyperref}
\usepackage{bookmark}
\usepackage[shortcuts]{extdash}
\usepackage{atlasphysics}
\usepackage{units}
\usepackage{amssymb}
\usepackage{tikz-feynman}
\usepackage{tabularx} 
 
%multi-row
\usepackage{multirow}
 


\catcode`\_=\active
\def_#1{\sb{\text{#1}}}

%------------------------------------------------------------------------------
%-------------------------    Angaben zur Arbeit   ----------------------------
%------------------------------------------------------------------------------

\author{Stella Oppermann}
\title{Optimization Studies for the Boosted Event Selection in a Search for Vector-Like Quark Pair Production in the Channels TT\texorpdfstring{$\longrightarrow$}~WbZt and BB\texorpdfstring{$\longrightarrow$}~WtZb with the ATLAS Experiment}
\date{Juli 2016} 
\birthplace{Greven}
\chair{Lehrstuhl für Experimentelle Physik IV}
\division{Fakultät Physik}
\thesisclass{Bachelor of Science}
\submissiondate{18. Juli 2016}
\firstcorrector{Prof. Dr. Kevin Kröninger}
\secondcorrector{Priv.-Doz. Dr. Reiner Klingenberg}

% tu logo on top of the titlepage
\titlehead{\includegraphics[height=1.5cm]{logos/tu-logo.pdf}}

\begin{document}
\frontmatter
%\input{content/hints.tex}
\maketitle

% Gutachterseite
\makecorrectorpage

% hier beginnt der Vorspann, nummeriert in römischen Zahlen
\input{content/00_abstract.tex}
\tableofcontents

\mainmatter
% Hier beginnt der Inhalt mit Seite 1 in arabischen Ziffern
\chapter{Motivation for a Search for Vector-Like Quarks}




\section{Short Overview of the Standard Model of Particle Physics}
Particle physics deals with elementary particles without any substructure and bound states of the elemantary particles forming other particles.
Moreover, the interactions between the particles  are studied.
Table \ref{standardmodel} shows the different elementary particles the Standard Model consists of.

\begin{table}
\centering
%\setlength{\tabcolsep}{3cm}
\begin{tabular}{|c|c c c|c|} 
\hline
 & $1^{st}$ generation & $2^{nd}$ generation & $3^{rd}$ generation &  Bosons \\
\hline
\hline
&	&	&	& \small{Vector-bosons}	\\
Quarks &  $\begin{matrix} u \\ \\ d \end{matrix}$ & $\begin{matrix} c \\ \\ s \end{matrix}$ & $\begin{matrix} t \\ \\ b \end{matrix}$ & $\gamma$ \\
& & & & g  \\
&	&	&	&	\\
& &  &  & $W^{\pm}$, $Z^{0}$ \\
Leptons & $\begin{matrix} \nu_{e} \\ \\ e \end{matrix}$ & $\begin{matrix} \nu_{\mu} \\ \\ \mu \end{matrix}$ & $\begin{matrix} \nu_{\tau} \\ \\ \tau \end{matrix}$ & \small{Scalar boson}  \\
&	&	&	&   H \\
&	&	&	&     \\
 \hline
\end{tabular}
\caption{Elemantary particles of the Standard Model.}
\label{standardmodel}
\end{table}



Both leptons and quarks are fermions with spin $s= \frac{1}{2}$. 
There are twelve fermions, with one anti-particle for each fermion, which are divided in three generations.     
For the leptons there is a lepton with charge $q = -1$ (electron e, muon $\mu$, tau $\tau$) and a corresponding neutrino without electric charge (electron-neutrino $\nu_{e}$, muon-neutrino $\nu_{\mu}$, tau-neutrino $\nu_{\tau}$) in each genaration.
The mass of charged leptons increases with the generation in which it is located and the neutrinos are supposed to be massless.\\
Quarks are elementary particles which cannot be observed in unbound states.
The quarks can be divided in up-type and down-types with charge $q = +\frac{2}{3}$ for the up- and $q = -\frac{1}{3}$ for the down-types.
Up-quark u, charm-quark c and top-quark t are up-types while down-quark d, strange-quark s and bottom-quark b are down-type quarks. 
The mass of the quarks increases with the generation as it was mentioned for the leptons.
The top-quark is of peculiar interest because it is the heaviest particle of the Standard Model and unlike every other quark does not occur in bound states.
Furthermore the Standard Model includes particles with spin 1 which are the gauge bosons.
They include the photon $\gamma$,the gluons g, the W bosons $W^{\pm}$ and the Z boson $Z^{0}$.
The gauge bosons are the mediators of the fundamental forces. 
The photon $\gamma$ mediates the electromagnetic, the gluons the strong  and the $W^{\pm}$ and $Z^{0}$ the weak force.  
Fermions and particles, which are bound states of the fermions, interact via the fundamental forces.
There are different characteristics particles have to fulfill to participate in the different interactions.
Particles with electric charge can interact via the electromagnetic force.
The strong interaction occurs between particles with colour charge, which are quarks and gluons themselves.
The weak force is mediated between particles with weak charge having a left-handed chirality.\\
The $\gamma$ and the gluons are massless, while the $W^{\pm}$ and $Z^{0}$ have a mass. 
W bosons are the only gauge bosons which have electric charge with +1 for the $W^{+}$ and -1 for the $W^{-}$.\\
Finally there is a scalar boson with spin 0, the higgs boson H.
The higgs boson carries mass but no electric charge.
It was predicted  by the higgs mechanism.
The mechanism explains the existence of mass for the elementary particles of the Standard Model.




\section{Beyond Standard Model Physics: Vector-like Quarks}
Vector-like quarks \cite{handbook} are part of  theories beyond the Standard Model. 
They are for example a necessary part of theories predicting the higgs as pseudo-Goldstone boson to explain why the mass of the higgs is as light as observed \cite{Th1,Th2,Th3}. 
Furthermore, they are part of the partial-compositeness theory of flavour in which they have an appearance as fermion resonances \cite{Th4,Th5}.
These ideas are used for example in little Higgs and composite Higgs models.\\
In addition to that, if the vector-like quarks will exist, they would offer the possibility to allow tree-level flavour-changing neutral currents and to break the GIM mechanism \cite{GIM}.
Moreover, new origins of CP violation would arise from the existence of vector-like quarks \cite{CP1,CP2}.\\
In the theories mentioned before, vector-like quarks  are predicted as fermions with spin $s = \frac{1}{2}$.
There are four types of vector-like quarks which are assumed to couple predominantly to the third generation Standard Model quarks.\\
The different vector-like quarks may appear in seven multiplets of which two are singulets, three doublets and two triplets.
Table \ref{vectorlikequarks} lists the different vector-like quark flavors with their specific charge and the possible multiplets they can be assigned too.

\begin{table}
\centering
\begin{tabular}{|c|c||c|} 
\hline
Flavor& Charge & Arrangement in multiplets\\
\hline
  & 	           & \multirow{3}{*}{$T^{0}_{L,R}$ , $B^{0}_{L,R}$ (singlets)}\\
T & $+\sfrac{2}{3}$ & \multirow{4}{*}{(X $T^{0})_{L,R}$,  ($T^{0} B^{0})_{L,R}$, ($B^{0} Y)_{L,R}$ (doublets)} \\
B & $-\sfrac{1}{3}$ & \\
Y & $+\sfrac{5}{3}$ & \multirow{3}{*}{(X $T^{0} B^{0} )_{L,R}$,  ($T^{0} B^{0} Y)_{L,R}$ (triplets)} \\  
X & $-\sfrac{4}{3}$ & \\
  &                &\\
\hline
\end{tabular}
\caption{This table contains the different types of vector-like quarks with their specific charge and the possible arrangement in multiplets. 
The zero superscript is used to denote the weak eigenstates and separate them from the mass eigenstates.
For the multiplets containing X and Y the mass and weak eigenstates are the same. }
\label{vectorlikequarks}
\end{table}

Unlike the Standard Model quarks coupling with a (V-A), which means a vector minus axialvector coupling structure regarding the weak interaction, the vector-like quarks have pure vector couplings.
Therfore, also vector-like quarks with right-handed chirality couple to the weak force.
Since the lagrangian describing the properties of the vector-like quarks is invariant under SU(2) transformations, explicit mass terms in the lagrangian are not forbidden because they do not break the local gauge invariance.
Hence, the vector-like quarks do not receive their masses from the higgs mechanism in contrast to the Standard Model quarks.\\
Because of the almost equal kinematic for the vector-like top and bottom quarks arranged in singulets, doublets and triplets the multiplets can be used similary for optimization studies \cite{8tevanalysis}.
Therfore mostly only singulets are considered .\\
Possible decay channels for the vector-like quarks include a combination of the vector bosons $W^{\pm}$, $Z^{0}$ or the higgs and a third generation quarks so that conservation of charge is guaranteed.\\
Vector-like quarks can be both pair and singly produced.
The main process for the pair production is the gluon fusion via the strong interaction.
The Feynman diagrams for the pair production process with the specific vector-like quark decay topology TT \texorpdfstring{$\longrightarrow$}~ZtWb and BB \texorpdfstring{$\longrightarrow$}~ZbWt are presented in figure \ref{feynmantts} for the vector-like top and \ref{feynmanbbs} for the vector-like bottom quark.
\begin{figure}[h!]
\centering
\includegraphics[width=11cm]{figures/TTS.pdf}
\caption{Feynman diagram for the vector-like top pair production process and the TT \texorpdfstring{$\longrightarrow$}~ZtWb decay channel.}
\label{feynmantts}
\end{figure}

\newpage

\begin{figure}[h!]
\centering
\includegraphics[width=11cm]{figures/BBS.pdf}
\caption{Feynman diagram for the vector-like bottom pair production process and the BB \texorpdfstring{$\longrightarrow$}~ZbWt decay channel.}
\label{feynmanbbs}
\end{figure}

In the following, the dilepton channel is considered, therefore all particles except the Z boson decay hadronically. 
The presented decay topology contains tree-level flavour-changing neutral currents.\\
Results from previous searches for pair production of vector-like T and B  in the di- and trilepton channel set mass limits for the vector-like top and bottom quark masses. 
The limits for the singlets are $m_{T} >$ \SI{780}{GeV} for the vector-like top and $m_{B} >$ \SI{830}{GeV} for the vector-like bottom quark \cite{masslimit}.  
This limits are determined with a confidence level of $95 \%$.







 











% kapitel2.tex
\chapter{The ATLAS Detector at the LHC}

\section{The LHC}
The LHC \cite{LHC} is a large hadron and lead nuclei accelerator and collider at the CERN.
It consists of four preaccelarators and the main accelarator which has a circumference of around \SI{27}{km}. 
In serveral stations of the main accelarator the ATLAS detector and other detectors are placed. 
In the main accelator both beams for the collision are accelerated in opposite direction at the same time.
For accelerated protons both bunches can contain up to $10^{11}$ protons.
Therfore each event includes many proton-proton collusions.
At each detector there is the possibility to bring the different beams together. 
The LHC currenlty reaches a center-of-mass energy of about $\sqrt{s} =$\SI{13}{TeV}. 
High center-of-mass energys are required for producing heavy particles and therefore essential for a search for very massive vector-like quarks.


\section{The ATLAS Detector}
The ATLAS Detector \cite{ATLAS} is a high luminosity experiment of the LHC. 
The construction of the ATLAS Detector is symmetric around the beam axis with cylindrical components and end caps to cover the full solid angle.
For discribing the ATLAS detector a specific coordinate system is generated.
The interaction point is the origin of the coordinate system and the z-axis is parallel to the beam axis while the x-y plane is perpenticular to the beam axis.
The azimuthal angle $\Phi$ is defined as angle around the beam axis while the polar angle $\Theta$ is measured starting from the beam axis. 
Moreover the pseudorapidity is defined according to $\eta = - ln(tan(\nicefrac{\Theta}{2}))$.
With this defined parameters distances in the pseudorapidity-azimuthal angle space can be calculated in the following way $\Delta R = \sqrt{\Delta \eta^{2} + \Delta \Phi^{2}}$.
Figure \ref{ATLAS} shows the cut-away view of the ATLAS Detector.
\begin{figure}[h!]
\centering
\includegraphics[width=11cm]{figures/atlas.jpg}
\caption{Cut-away view of the ATLAS detector with the different constituents.}
\label{ATLAS}
\end{figure}
There are four main parts the ATLAS detector consists of.
To begin with there is the inner detector which is the first component around the interaction point.
The task of the inner detector is the vertex resolution and furthermore the resolution of the momenta for all charged particles.
It consits of pixel and silicon microstrip detectors trackers and a Transition Radiation Tracker.
A solenoid magnet surrounds the inner detector and immerses it in a magnetic field. 
This causes the curvature of the charged particles tracks and ensures the momentum measurement.\\
The solenoid magnet is surrounded by the electromagnetic and hadronic calorimeters.
The electromagnetic calorimeter is further inside and makes precision measurements of electron and photon energies possible.
Electrons and photons ideally deposit all their energy in the electromagnetic calorimeter by interacting with the detector material.\\
The hadronic calorimeter performs the same goal as the electromagnetic calorimeter for hadrons, which are bound states of the quarks.
Hadrons form jets in the calorimeters.
The particles loose their energy because of hadronic showers in the hadronic calorimeter.
The outermost component of the ATLAS detector is the muon system or muon spectrometer.
For the ideal case only muons reach the muon spectrometer.
The muon spectrometer is immersed with a magnet field resulting from toroid magnets.
The magnets causes the deflection of the muons and enable momentum measurements as mentioned for the inner detector.





% kapitel3.tex
\chapter{Monte-Carlo Samples}

\section{Background Processes}
\label{samples}
In order to perform an unbiased analysis in particle physics, the analysis strategy is studied with simulated samples and only later on compared to data.
In this analysis, only simulated data are considered.
The simulated background samples contain background processes generated with Monte Carlo simulations.
For the background,  Monte Carlos samples uses random number generators in combination with branching ratios predicted in the Standard Model to simulate processes.
The Monte Carlo samples used in this analysis were passed through a full detector simulation of the ATLAS detector \cite{simulation1, simulation2}.\\
The background is divided into five parts, considering the dominant background processes separately and combining the other backgrounds in one category.
There are the Z+jets backgrounds, which include Z+bottom, Z+charm and Z+light processes.
The Z+bottom process is composed of a Z candidate and a number of b-jets, both are produced at the same time.
The other Z+jets backgrounds are composed similarly, with a number of charm-jets for Z+charm and a number of light-jets (strange-,up-,down-jets).
Moreover, there is the \ttbar{} background, which includes a produced top-quark pair, and the category ``other background'', which contains for example diboson processes, $t \bar{t} e^{+}\!e^{-}$ and $t \bar{t} \mu^{+}\!\mu^{-}$ processes.
The diboson processes include a produced pair of vector bosons (WW, ZW, ZZ) and the $t \bar{t} e^{+}\!e^{-}$ respectively $t \bar{t} \mu^{+}\!\mu^{-}$ processes a associated production of a top-quark pair and a pair of electrons or muons.
Since the different processes and the signal can lead to simular signatures in the detector, caused by partially equal final state products, the processes are considered as main background. \\
The various processes are simulated with different event generators.
Event generators simulate the processes starting from the pp-collision and seperate them into the following stages:
Production of heavy particles, radiation of lighter particles (gluons, photons), decay of heavy unstable particles, hadronization and the decays of the particles resulting from the hadronization into long-lived particles. 
The resulting particles finally enter the detector simulation.
There are event generators, which only generate until the hadronization process and therefore shower programs are added.\\
To simulate the Z+jets samples, the event generator Sherpa 2.2 is used.
Sherpa \cite{Sherpa} performs both the simulation of processes until the hadronization and beyond the hadronization.
Sherpa simulates at leading order precision in $\alpha_{s}$. 
For the \ttbar{} background processes, the generator Powheg is utilized with Pythia 8.1 \cite{Pythia} as shower program.
Powheg \cite{Powheg} is a next-to leading order event generator.
For the diboson background, there are both simulated samples with Sherpa 2.2 and Powheg + Pythia 8.1.
The processes $t \bar{t} e^{+}\!e^{-}$  and $t \bar{t} \mu^{+}\!\mu^{-}$ are simulated with Madgraph as event generator and Pythia 8.1 as shower program.
Madgraph \cite{Madgraph} is a event generator which simulates at leading order.\\
In the analysis the samples are normalized to the value of the integrated luminosity $\int{L} \mathrm{d}t = 3.2 \mathrm{fb}^{-1}$ to provide the possibility of comparing the simulated data with the measured data.\\
In the following analysis, the weighted number of events denote the expected number of events considering the integrated luminosity $\int{L} \mathrm{d}t = 3.2 \mathrm{fb}^{-1}$.


\section{Signal Processes}
The signal also includes processes generated from Monte Carlo simulations. 
In this analysis, both vector-like top and bottom quark are considered to be arranged in the singlet.
The processes are referred to as TTS for the vector-like top quark pair production and BBS for the vector-like bottom quark pair production.
Both TTS and BBS are simulated with the event generator Protos and Pythia 8.1 as shower program.
Protos \cite{Protos} is a leading order event generator.\\
As mentioned in section \ref{samples} for the backround, the samples for the signal are also normalized to the value of the integrated luminosity $\int{L} \mathrm{d}t = 3.2 \mathrm{fb}^{-1}$.
Therefore the weighted number of events also represents the expected number of events for the integrated luminosity $\int{L} \mathrm{d}t = 3.2 \mathrm{fb}^{-1}$.

% kapitel4.tex

\chapter{Preselection in the Boosted Channel}

\section{Event Preselection}
\label{Event Preselection}
The Monte-Carlo samples used in this analysis have a specific preselection.\\
In table \ref{Event preselection} the different requirements are listed, which are part of this preselection. 

\vspace{0.5cm}

\begin{table}
\centering
\setlength{\tabcolsep}{3cm}
\begin{tabular}{|c|} 
\hline
\textbf{Event preselection} \\
\hline
\hline
\vspace{-0.3cm}
\\
Z boson candidate preselection:\\
\vspace{-0.1cm}
\footnotesize{$\bullet$ pair of opposite sign leptons (same flavor)} \\

\footnotesize{$\bullet$ $\mid m_{l^{+} l^{-}} - m_{Z} \mid$ < \SI{10}{GeV}} \\
\\
\vspace{-0.9cm}
\\
\hline
\vspace{-0.3cm}
\\
$\geq 2$ central jets \\
\vspace{-0.4cm}
\\
$\geq$ 2 b-tags\\
\vspace{-0.4cm}
\\
= 2 leptons \\
\hline
\end{tabular}
\caption{Preselection criteria.}
\label{Event preselection}
\end{table}


The first aspect is the Z boson candidate preselection. 
According to the studied leptonic decay channel of the Z this includes a reconstructed Z candidate mass with \SI{10}{GeV} around the real Z pole resulting from two opposite sign leptons with same flavor.
In this analysis only electrons and muons are considered in the Z candidate reconstruction and not the $\tau$ lepton or neutrinos.
Figure \ref{Zmass} shows the distribution of the reconstructed Z mass. 
Both signal and background are presented in one plot and are normalized to unity. 
The background is illustrated as colour filled histogram with a specific colour for each background process which is listed in the legend.
The different background processes are stacked. 
The signal is shown as solid red and blue line for the BBS and TTS process.
In the legend the weighted number of events is displayed in brackets next to each process.
Furthermore the statistic errors are shown as vertical lines.
The shape of the distribution looks like expected because it displays a pole mass distribution which is smeared because of detector influences.
Processes including a real Z like the signal and Z+jets processes have a peak at the Z mass. 
The \ttbar{} background process has no peak at the Z pole mass which is as anticipated as well because the leptons result from W boson decays.\\
Furthermore in the preselection at least two central jets with $\mid\eta\mid < 2.5$ and $p_{T} < \SI{25}{GeV}$ are required.
These jets are called central jets.
The $p_{T}$ requirement should ensure that the jets are calibrated and the $\eta$-regio is limited because the detector resolution in this $\eta$-region is higher than in regions with $\mid \eta \mid > 2.5$.  
Because of the specific decay topology two or more b-tags for the central jets are expected.
One b-jet results straight from the vector-like quark decays T\texorpdfstring{$\longrightarrow$}~Wb and B \texorpdfstring{$\longrightarrow$}~Zb.
The other is caused by the third generation top quark decays t\texorpdfstring{$\longrightarrow$}~Wb.
The top quarks are the decay products of the second vector-like quarks with T\texorpdfstring{$\longrightarrow$}~Zt and B\texorpdfstring{$\longrightarrow$}~Wt.\\  
The last aspect of the preselection listed in table \ref{Event preselection}  is the requirement of exact two leptons. 
This should guarantee the analysis in the dileptonic channel. 

\begin{figure}
\centering
\includegraphics[width=11cm]{figures/Zmass.png}
\caption{Distribution for the reconstructed Z boson mass after the preselection.}
\label{Zmass}
\end{figure}

\section{Boosted Analysis Strategy}
The samples are simulated with a vektor-like quark mass of $m_{T/B} =$ \SI{900}{GeV}. 
Because of the massive vektor-like T and B the decay products have very high transverse momenta. 
Caused by the high transverse momenta the particles of the subsequent decays are strongly collimated. 
This situation is illustrated in figure \ref{boosted} for a top quark decay.
The figure shows the decay for a low and a high top $p_{T}$.
As mentioned before the figure shows that the decay products of the top are collimated for a high top $p_{T}$.\\
This collimated structure of the particles have an inpact on the clustering algorithms used in the ATLAS experiment.
Jets can for example be clustered with a anti-$k_{t}$ jet clustering algorithm \cite{antikt}.  
The anti-$k_{t}$ algorithm clusters soft particles around a hard particle within a given radius and forms conical jets. 
If two hard particles are located within an area closer than $\Delta R$ the clustering algorithm is not able to form a smooth circular shape around the hard particles.  
For a low top $p_{T}$ a $\Delta R = 0.4$ can be used.
In the high $p_{T}$ regions it is probable, that the jets from the top decay, which are clustered with $\Delta R = 0.4$  merge because the one decay product can be in an area $\Delta R$ around the other. 
Hence as mentioned before the anti-$k_{t}$ algorithm is not able to form smooth circular shapes. 
To avoid this problem the boosted analysis strategy can be used for high top transversal momenta.
The idea of the boosted analysis strategy is to cluster one jet which have all decay products of a boosted particle in it.
The area, where the decay products of a particle are located can be estimated with the formula $R \approx \frac{2m}{p_{T}}$ .
In the boosted analysis a radius of $\Delta R = 1.0$ for the cluster algorithm is used. 
A combined result from the top quark mass measurement is $m_{t} =$ \SI{173.29 \pm 0.95}{GeV} \cite{topmass}.
Using the formula mentioned before the boosted analysis is sensibel for a $p_{T}$ of the top quark about \SI{346}{GeV} and more. 
The boosted analysis strategy can be used for a boosted decay of a W boson, too.

\begin{figure}
\centering
\includegraphics[width=11cm]{figures/boost.png}
\caption{Representative figure for a top quark decay with a high and a low top $p_{T}$ created by Emily Thompson.}
\label{boosted}
\end{figure}

In this analysis jets clustered with  $\Delta R = 1.0$ are used.   
These jets are called large-R jets. 
Large-R jets offer the possibility to introduce top- and boson tagging.
Table \ref{calibration} contains different conditions large-R jets have to fulfill to ensure that they are calibrated.
Only calibrated large-R jets are considered. 

\newpage

\begin{table}[h!]
\centering
\setlength{\tabcolsep}{3cm}
\begin{tabular}{|c|} 
\hline
\textbf{Large-Rjet only calibrated for:}\\
\hline
\hline
$P_{T}$(large-R jet) $\geq \SI{200}{GeV}$\\
$\mid \eta \mid < 2.0$ \\
m(large-R jet) $> \SI{50}{GeV}$\\
\hline
\end{tabular}
\caption{Calibration criteria for large-R jets.}
\label{calibration}
\end{table}


Furthermore large-R jets within an area of $\Delta R < 1.0$ around the Z candidate reconstructed from two electrons are discarded in this analysis.
Figure \ref{deltaR} shows the distribution of the $\Delta R$ between all large-R jets in an event and the Z candidate of this event.
The distribution has a clear peak at $\Delta R = 0$, which illustrates that the decay products of the Z candidate are often misidentified as large-R jets.
This is caused by the electrons which can be misidentified as large-R jets because of a simular signature in the detector.
Because of that it is reasonable to ignore the large-R jets within an area of $\Delta R < 1.0$ around the Z candidate reconstructed from two electrons to avoid that they are treated as large-R jets in the analysis.
This requirement is called overlap removal.

\begin{figure}
\centering
\includegraphics[width=10cm]{figures/deltaR.png}
\caption{Distribution for the $\Delta R$ of the large-R jets and the Z boson candidate after the preselection.}
\label{deltaR}
\end{figure}

      



\section{Basic Selection Implied by the Boosted WbZt Topology  }
The decay topologys TT \texorpdfstring{$\longrightarrow$}~ZtWb and BB \texorpdfstring{$\longrightarrow$}~ZbWt for the optimization studies in this analysis are defined as mentioned before. 
Because of the designated products of the vector-like quark decays further selection can be added to the preselection discussed in section \ref{Event preselection}, which is also used in more general studies without determining specific decay channels.       
It is reasonable that at least two large-R jets should be required considering that there is one top quark and one W boson, which could lead to large-R jets. 
Events without at least one top- and W-tag should also be rejected.  
Figure \ref{ljetmult} shows the multiplicity of the large-R jets.

\begin{figure}
\centering
\includegraphics[width=10cm]{figures/multipliljet.png}
\caption{Plot for the large-R jet multiplicity after the preselection.}
\label{ljetmult}
\end{figure}

As expected the background processes mostly have no large-R jets because in most cases the $p_{T}$ of the particles is not high enough to produce large-R jets.
The decay topology considered in this analysis give reason to expect two large-R jets resulting from the boosted top Quark and W boson decays.
Therefore signal distribution looks not like expected caused by the fact that about 40 \% of the events only have one large-R jets. 
An explanation for the distribution could be, that the W bosons from T \texorpdfstring{$\longrightarrow$}~Wb and B \texorpdfstring{$\longrightarrow$}~Wt  could be located in an area of $\Delta R = 1.0$ around the Z candidate and therefore could be ignorated caused by the overlap removal mentioned earlier.
For the BBS signal process the case mentioned before is also possible for the jet resulting from the top quark.
Furthermore for the TTS signal process it is possible that the decay products of the W-boson and the top quark are clustered in one large-R jet because they are in an area of $\Delta R = 1.0$.\\
The requirement of two large-R jets selects the background from the signal and rejects a lot of background.
There are also a lot of signal events which are ignored because of the problematic mentioned before.
A representation for the distribution of the top- and W-tag multiplicity can be found in  figure \ref{topmultipli} and \ref{bosonmultipli}.

\newpage


\begin{figure}[h!]
\centering
\includegraphics[width=11cm]{figures/topmultiplicity.png}
\caption{Plot for the top-tag multiplicity after the preselection.}
\label{topmultipli}
\end{figure}

\begin{figure}[h!]
\centering
\includegraphics[width=11cm]{figures/Wmultiplicity.png}
\caption{Plot for the W-tag multiplicity after the preselection.}
\label{bosonmultipli}
\end{figure}

The plot for the top-tag multiplicity shows that most background events don't have a top-tagged large-R jet.
That is sensible if the Z+jets backgrounds are regarded because in these processes no top quark is included. 
The \ttbar{} process should also have no top-tag because the W bosons produced by the top quarks decay in the leptonic decay channel because there are two leptons required.
Therefore only  b-tagged jets are produced.
If the leptons resulting from the Z for the Z+jets backgrounds are not in an area $\Delta R = 1.0$ around the Z candidate they are not rejected by the overlap removal.
For the \ttbar{} process it is also possible that the leptons from the W boson decays are counted as large-R jets.
Hence there are some events with one top-tag in the background distribution.\\
The signal distribution looks not like assumed because there are a lot of events having two top-tagged large-R jets.
This is caused by the fact that the top-tagger also tag the large-R jet resulting from the W-boson as top-tag.\\
BEMERKUNG : vielleicht verteilung W-tagged jets die nicht getop-tagged sind in den Anhang zum aufzeigen?.\\
The background distribution for the W-tagged large-R jets can be explained with the same argumentation as for the top-tag multiplicity.
For the signal processes there is a high amount of events without W-tagged large-R jet ( about 80 \% for TTS and 65 \% for BBS), which is not like expected.
This could be caused by the argumentation mentioned before that the large-R jet pruduced by the W-boson is ignored because of the overlap removal.
In addition to that the boson-tagger doesn't work perfectly but has an efficiency of 50 \%.\\
After demanding all three parts of the basic selection implied by the boosted WbZt topology a lot of background is rejected which becomes obvious in the three discributions discribed before.
The unweighted number of events for signal and background after the basic selection is listed in table \ref{numberoevents}.
\vspace{-0.19cm}

\begin{table}
\centering
%\setlength{\tab}{\textwidth}
\begin{tabular}{|c|c|c|} 
\hline
\textbf{Process} & \textbf{Unweighted number of events} & \textbf{Weighted number of events}  \\
\hline
\hline
Z+light & 3 & 0.0005\\
Z+charm & 2 & 0.0060\\
Z+bottom & 77 & 0.1680\\
ttbar & 6 & 0.3739\\
Other BG & 1331 & 0.3298\\
TTS & 214 & 0.1094\\
BBS & 176 & 0.1089\\
\hline
\end{tabular}
\caption{Unweighted and weighted number of events for signal and background processes.}
\label{numberoevents}
\end{table}


The listed numbers reveal that after the event preselection and basic selection there are for both signal and background very  low numbers of events.
An optimization study without enough statistic is not convincing because the distributions are containing too much error.
Therefore the optimization study in this analysis is performed without requireing at least one W-tag.
The statistic after the basic selection is much higher for both signal and background without this requirement.

 



















% kapitel5.tex
\chapter{Optimization Studies for the Event Selection}

\section{Search for Discriminating Variables }

\section{Significance for Different Cuts}

\section{Proposel for a Boosted Event Selection}


% kapitel6.tex
\chapter{Summary and Conclusions}



\appendix
% Hier beginnt der Anhang, nummeriert in lateinischen Buchstaben
\input{content/a_anhang.tex}

\backmatter
\printbibliography

\cleardoublepage
\input{content/eid_versicherung.tex}
\end{document}
